% template for Phoebe
%
\documentclass[language=english]{phoebe}

% ---------------------------------------------------------------------
% Place your own packages here 
% ---------------------------------------------------------------------

\usepackage{blindtext}

% ---------------------------------------------------------------------
% Place your own commands here 
% ---------------------------------------------------------------------


% ---------------------------------------------------------------------
% Define some variables
% ---------------------------------------------------------------------

\title[\LaTeX\ template \& style guide]{\phoebe\ -- Gazette for student physics: \\\LaTeX\ template \& style guide} 
\author[F. Scheuermann \& C. Otte]{
Fabian Scheuermann,$^{1\dag}$
Christoph Otte$^{2}$
\\
$^{1}$Astronomisches Rechen-Institut, Zentrum f\"{u}r Astronomie der Universit\"{a}t Heidelberg, M\"{o}nchhofstra\ss e 12-14, 69120 Heidelberg, Germany, \\
$^{2}$Universitätsbibliothek Heidelberg, Plöck 107-109, 69117 Heidelberg, Germany
}

%
\email{f.scheuermann@uni-heidelberg.de}
\doi{}
\date{Accepted XXX. Received YYY; in original form ZZZ}
\pubdate{2022.09.23}

 % load file with bibliography
\addbibresource{paper.bib}  


% ---------------------------------------------------------------------
% The main Document
% ---------------------------------------------------------------------

\begin{document}


% ---------------------------------------------------------------------
% Frontmatter
% ---------------------------------------------------------------------

\thispagestyle{plain}
\maketitle

\begin{abstract}
Every article is required to include an abstract. This should be a short summary of the content with the most important results. It should be no longer than 250 words. This document is is part of the official \LaTeX\ template for Phoebe. Included are a few examples on how to include equations, figures or tables.
\end{abstract}

\begin{keywords}
Journal -- typography -- \LaTeX 
\end{keywords}

% ---------------------------------------------------------------------
% Main Body of the article
% ---------------------------------------------------------------------


\section{Introduction}
Phoebe is an open access journal that aims to give physics students the opportunity to document their personal gain in knowledge for themselves and others. Many exciting discussions, e.g. during breaks of lectures, which allow a deeper understanding, are usually only caught by a small part of the students. Until now, there has unfortunately been no way to record these insights in the long term and make them available to others. This is where Phoebe wants to start and promote a broader discourse.

This is the official \LaTeX\ template for the \emph{Phoebe Gazette}\footnote{\url{www.phoebe-gazette.de}}. A copy can be obtained from
\begin{center}
    \url{https://github.com/phoebe-gazette/LaTeX-template}
\end{center}
We recommend to use \href{https://www.overleaf.com?r=4714e231&rm=d&rs=b}{overleaf.com} to edit your \LaTeX documents. Here is a small example on how to use it
\begin{lstlisting}
\documentclass{phoebe}

% define some variables
\title[Running title]{The main title} 
\author[Running author]{John Doe}
\doi{will be filled out by the journal}
\pubdate{will be filled out by the journal}

% file with references
\addbibresource{paper.bib}  

\defabstract{A short summary of the content.}

\begin{document}

% your text goes here

\end{document}
\end{lstlisting}

This document is organised as follows: Section~\ref{sec:typography} gives some general typography rules and Section~\ref{sec:latex} provides some examples on how to use this class. 

\section{Frontmatter}

The titlepage with title, author list and abstract is handled by the document class. Simply change the commands from this template.


\section{Typography}\label{sec:typography}

% we follow the Monthly Notice style guide
% https://academic.oup.com/mnras/pages/general_instructions?login=true#6%20Style%20guide

This section provides some guidelines on how the article should be formatted. Many of them are already implemented in the class file and the authors do not need to worry about them.

\subsection{Layout}\label{sec:layout}

The article should be typesetted with \LaTeX\ in a two-column layout. Use \verb|\section{}| and \verb|\subsection{}| to structure your document. 

\subsection{Hypen and dash}
A hypen (single \verb|-|) is used to combine words (e.g.~low-density). The en-dash (two \verb|--|) is slightly larger and is used to indicate ranges (e.g.~\SIrange{2}{10}{\kg}). The en-dash is identical in length to the minus sign. When in math mode, \LaTeX will automatically use a minus sign when a single \verb|-| is used. em-dashes (\verb|---|) should not be used in the journal.

\LaTeX assumes that a period marks the end of a sentence and as such puts a bit of extra space after it. This is wrong if the period is used in an abbreviation, e.g.\ i.e. To avoid this, place a space (\verb|e.g.\ |) after the period. In the previous example, the abbreviation period also marks the end of the sentence. In such case only one period is required. 

\subsection{Quotes}
It is strongly recommended to make use of \texttt{csquotes} as \enquote{This package provides advanced facilities for inline and display quotations}. Should you decide to write your article in english, you should load the document class with the \texttt{english} option.

\begin{lstlisting}
\documentclass[language=english]{phoebe}
\end{lstlisting}


\subsection{Math mode}

Vectors are are set in bold (no arrow). 

A one-line formula should use the \texttt{equation} environment
\begin{equation}\label{eq:example}
    y = \int_1^\infty \frac{1}{x^2}\, \mathrm{d}x.
\end{equation}
And equations should always be punctuated. If the formula consists of multiple lines, the \texttt{align} environment enables aligning the equations
\begin{align}
    y &= \int_1^\infty \frac{1}{x^2}\, \mathrm{d}x \\
    &= \left. -\frac{1}{x}\right|_1^\infty \\
    &= 0 + 1 = 1
\end{align}
The differential of the variable is written in roman (\verb|\mathrm{d}|) and not italics.

\subsection{Units}

Variables are set in italics (this happens automatic in math mode), however units are always roman (upright). They should be separated by a non-breaking space. For reciprocal units, use a superscript, e.g.\ $\si{\cm\per\s}$. Use either SI or cgs units.

Please make use of the \texttt{siunitx} package (already loaded with this class) as it takes care of the aforementioned rules. If you need a unit that is not already defined, you can define your own units like so
\begin{lstlisting}
\DeclareSIUnit\parsec{pc}
\end{lstlisting}

The axis of plots must have the according unit. This should be written as $x\,/\,\si{\cm}$. Note that commonly found notation $x\,[\si{\cm}]$ is not acceptable. Square brackets denote the unit of a quantity (just like value is denoted by curly brackets), i.e.\ $x=
\{x\}[x]$. Take the example $x=\SI{12}{\cm}$, i.e.~$[x]=\si{\cm}$. For more details, see section 7 of the \href{https://www.nist.gov/pml/special-publication-811/nist-guide-si-chapter-7-rules-and-style-conventions-expressing-values}{NIST Guide to the SI}.

\section{\LaTeX}\label{sec:latex}

Here are a few tips and tricks on how to use \LaTeX.

\subsection{Figures and Tables}

In Figure~\ref{fig:example} and Table~\ref{tbl:example} we show an example for a figure and a table respectively. 
\begin{figure}
    \centering
    \includegraphics[width=0.8\columnwidth]{tex}
    \caption{This is an example for a figure. The caption is below the image. Credit: \url{https://tug.org/}.}
    \label{fig:example}
\end{figure}

For tables, they should have no vertical lines. The caption for tables should be above the table.
\begin{table}
    \centering
    \caption{This is an example for a table. The caption is above the table.}
    \begin{tabular}{cc}\toprule
        Column 1  & Column 2 \\\midrule
        1 & A \\
        2 & B \\
        3 & C \\\bottomrule
    \end{tabular}
    \label{tbl:example}
\end{table}

\subsection{Examples for code}

To include code in a text, the \texttt{verbatim} environment is often used to set the text. This sets the exactly like it is typed, i.e.\ it ignores \LaTeX\ commands. A better way for tpyesetting code is with the \texttt{listings} package. It supports a number of programming and provides code highlighting for them.
\begin{lstlisting}[language=Python]
import numpy as np
import matplotlib.pyplot as plt

fig,ax=plt.subplots()

x = np.linspace(0,2*np.pi)

for i in np.range(1,4):
    y = np.sin(i*x)
    ax.plot(x,y)
plt.show()
\end{lstlisting}

\subsection{Citations and references}
The citations and bibliography should use the \emph{Harvard author (year)} style. 
This class uses \texttt{biblatex} and \texttt{biber} to manage the bibliography. Therefore, to cite another paper in the text use \verb|\textcite| which produces \textcite{Scheuermann+2022} or to cite one in parentheses use \verb|parencite| which yields \parencite{Kreckel+2020}. You can also use the \texttt{natbib} commands (even though natbib is not used) like \verb|\citep| which results in \citep{Emsellem+2022,Lee+2022,Leroy+2021b}.

I would recommend \textsc{jabref} to manage your \verb|.bib| file.

Using \verb|\label{}| and \verb|\ref{}| to cross-reference within the paper is preferred to explicitly writing out the references as \texttt{hyperref} will create links. 
References in the text should be written as Figure~\ref{fig:example}, Table~\ref{tbl:example} and Equation~\ref{eq:example}.


\section{Summary}
Happy \TeX{}ing

% ---------------------------------------------------------------------
% Backmatter
% ---------------------------------------------------------------------


\printbibliography[title = {References}]

% ---------------------------------------------------------------------
% End of the Document
% ---------------------------------------------------------------------

\bsp	% typesetting comment
\end{document}
